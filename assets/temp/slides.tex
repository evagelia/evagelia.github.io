\documentclass[hyperref={pdfpagelabels=false}]{beamer}
\usepackage{etex}

%If we want "clean" slides without the "pauses"
%\documentclass[handout]{beamer}

\mode<presentation>
\usepackage{lmodern}

\usepackage{thmtools}
\usepackage{thm-restate}

\usetheme{Malmoe} 
\definecolor{myBlue}{RGB}{1,69,124}
\usecolortheme[named=myBlue]{structure}
\beamertemplatenavigationsymbolsempty % To get rid of nav buttons

\usepackage{array}
\usepackage{bbm} 					%For nice expected value symbol
\usepackage{mathtools}			%For xRightarrow
\usepackage[ruled]{algorithm2e}	%For "algorithm" environment
\usepackage{bm}					%For bm
\renewcommand{\thealgocf}{} %no remove number from algorithm caption

\usepackage{tikz}					%Tikzzzzzzzzzzzzzz <3
\usetikzlibrary{shapes.misc}
\usetikzlibrary{arrows.meta, shapes.geometric, calc, shadows, decorations.pathreplacing} %tikz stuff
\usepackage{multirow,bigdelim}
\usepackage{enumerate}
%%%%%%% New Commands %%%%%%%%
\newcommand{\lsum}{\displaystyle \sum}		%To display large sum sign

%Proof Idea
\newenvironment{proofid}{%
  \renewcommand{\proofname}{Proof Idea}\proof}{\endproof}

%Proof Sketch
\newenvironment{proofsk}{%
  \renewcommand{\proofname}{Proof Sketch}\proof}{\endproof}

%To make theorems and lemmas numbered
% \setbeamertemplate{theorems}[ams style] %This makes Lemma/Theorem etc to appear bold
 \setbeamertemplate{theorems}[numbered]
%%-----------------------------------------------------------------------------
% Fonts
%%-----------------------------------------------------------------------------
%\usepackage{fontspec}
\usepackage{amsthm}

%\setmainfont{CMU Serif}
%\setsansfont{CMU Serif} 
%\setmonofont{CMU Serif} 

%%-----------------------------------------------------------------------------
% Beamer Stuff
%%-----------------------------------------------------------------------------
\usepackage{hyperref}
\usepackage{adjustbox}
\usepackage{amsfonts,amssymb,wasysym}
\usepackage[accumulated]{beamerseminar}
\usepackage{beamertexpower}
\usepackage{amssymb}%for checkmark in figure
\usepackage{pifont}%for nice xmark in figure

\renewcommand*\insertshorttitle{\secname}%\hfill%

%%% Parameters %%%
\newcommand{\authorname}{Evangelia Gergatsouli, Brendan Lucier, Christos Tzamos}
\newcommand{\mytitle}{The Complexity of Black-Box Mechanism Design with Priors}

\title[]{\mytitle}

\author[E.Gergatsouli, B.Lucier, C.Tzamos]{\emph{Evangelia Gergatsouli}$^1$, Brendan Lucier$^2$, Christos Tzamos$^1$}

\institute{ 
$^1$University of Wisconsin-Madison\\
$^2$Microsoft Research\\
}
\date{ACAC, Athens, Greece, August 2019}

%%%%%%%%%%%%%%%%%%%%%%%%%%% MY NEW COMMANDS %%%%%%%%%%%%%%%%%%%%%%%%%
\renewcommand{\footnotesize}{\scriptsize}
%Add numbering of slides...
\addtobeamertemplate{navigation symbols}{}{%
    \usebeamerfont{footline}%
    \usebeamercolor[fg]{footline}%
    \hspace{1em}%
    \insertframenumber/\inserttotalframenumber
}

%Useful bold not italic values vector
\newcommand{\refscolorz}{teal} %References color
\renewcommand{\v}{\text{\textbf{v}}}
\newcommand{\xmark}{\ding{55}} %needed for figure
\newcommand{\prob}[2]{\textbf{Pr}_{#1} \lp[ #2 \rp]}
\newcommand{\poly}{\text{poly}}
\newcommand{\E}{\mathbb{E}}
\newcommand{\ind}[1]{\mathbbm{1}{ \{ #1 \} }}

\newcommand{\mybullet}{$\textcolor{myBlue}{\blacktriangleright}$ } %Bullet same as itemize, for text
\newcommand{\myspace}{0.5cm}
\renewcommand{\sp}{\mbox{} \hspace{\myspace}}	%mlkies spacing for prev results slide
\newcommand{\spitem}{\mbox{} \hspace{1cm}}	%mlkies spacing for prev results slide

\newcommand{\eps}{\varepsilon}
\newcommand{\reals}{\mathbb{R}}
\newcommand{\lp}{\left}
\newcommand{\rp}{\right}

\newcommand{\A}{{A}}
\newcommand{\Wel}{\textsc{Wel}}

\newcommand\Dist{\mathcal{D}}
\newcommand\Alloc{\mathcal{F}}
\newcommand\Algo{\mathcal{A}}
\newcommand\Pay{\mathcal{P}}
\newcommand\Mech{\mathcal{M}}
\newcommand\Domain{\mathcal{X}}
\newcommand\Range{\mathcal{Y}}
\newcommand\Exp{\mathbb{E}}

\declaretheorem[name=Lemma]{lem}


%%TO remove "figure" in figure's captions
%\setbeamertemplate{caption}{\raggedright\insertcaption\par}
\setbeamertemplate{headline}{}

\begin{document}

\begin{frame} %title
\titlepage
\end{frame}

\begin{frame}
\tableofcontents
\end{frame}

\section{Introduction}

\begin{frame}
	\frametitle{Mechanism Design for Welfare}
	
	 \begin{table}[H]
		\begin{tabular}{p{0.5\textwidth}  p{0.4\textwidth} }

	\begin{figure}[!h]
			\centering
				\begin{tikzpicture}[scale=0.70]

\tikzset{VM/.style={shape=rectangle}}
\tikzset{Client/.style={shape=circle}}

\node[Client]  (client1) at (3,3) {\includegraphics[width=0.9cm]{bob.png}} ;
\node[Client]  (client2) at (7,3) {\includegraphics[width=1.3cm]{patrick.png}} ;

\foreach \i in {1,2,3,4}{
		\node[VM] (vm\i) at (2*\i,-0.5) {\includegraphics[width=0.8cm]{patty.jpg}};
	\node[] at (2*\i,-1) {};%$\text{Item}_\i$};
}

%Draw fesible allocation
\onslide<2>{
	\draw[->, thick] (client1) -- (vm1); 
	\draw[->, thick] (client2) -- (vm2); 	
	\draw[->, thick] (client2) -- (vm3); 	
	\draw[->, thick] (client2) -- (vm4); 	
 };

\onslide<3> {
	\draw[->, thick] (client1) -- (vm1); 
	\draw[->, thick, red] (client2) -- (vm2); 	
	\draw[->, thick, red] (client2) -- (vm3); 	
	\draw[->, thick, red] (client2) -- (vm4); 	
};

\onslide<4,5> {
	\draw[->, thick] (client1) -- (vm1); 
	\draw[->, thick] (client1) -- (vm2); 	
	\draw[->, thick] (client2) -- (vm3); 	
	\draw[->, thick] (client2) -- (vm4); 	
}

\onslide<5> {
	\node[] at (1.5,3) {$\$42$};
	\node[] at (8.5,3) {$\$13$};
 };

\end{tikzpicture}

		\end{figure}
& 
 \begin{itemize}
 		\item $n$ agents, $m$ items
		\item Agent $i$ has \emph{private} value $\text{v}_i(S)$ for set $S$ of items
%		\item Agents report $\text{v}'_i$ 
		\item Feasibility constraint $\Alloc$ (e.g. at most 2 items per agent)
		\item Goal: allocate items to maximize welfare
 \end{itemize}
			 \end{tabular}
	 \end{table}

\vspace{-2em}
Mechanism $\Mech$: Given reported values $\v'$ decide:
\onslide<2,3,4,5>{
		\begin{itemize}
				\item Allocation rule $\Algo$: how to give out items 
%						(feasible allocation $\Algo$)
						\onslide<5>{	\item Payment rule $\Pay$: Who pays what}
			\end{itemize}
	}

\end{frame}

\begin{frame} 
	\frametitle{Black-Box Mechanism Design}
	
	Algorithmic problem: Find allocation rule $\Algo$ that maximizes welfare\\	
	Can we turn this into truthful mechanism?\\
\pause
	\begin{figure}[!h]
		\centering
			\begin{tikzpicture}[scale=0.55]
%Style macros
\tikzset{square/.style={black, thick}}

%Macros for graph parameters (length/height/radii)
\pgfmathsetmacro{\length}{6}
\pgfmathsetmacro{\height}{4}
\pgfmathsetmacro{\x}{0}
\pgfmathsetmacro{\y}{0}

%%% MAIN CODE %%%
%Big square, mechanism
\draw[square] (\x, \y) rectangle (\x + \length,\y + \height) node[xshift=-0.5cm, yshift=-0.2cm]{$\Mech$};

%Small square, algo
\draw[square] (\x+0.4*\length, \y+ 0.4*\height) rectangle (0.8*\length,0.8*\height) node[xshift=-0.3cm, yshift=	-0.15cm]{$\Algo$};
	
\draw[->] (\x+\length,0.5*\y+0.5*\height) -- (\x+\length + 1.5, 0.5*\y + 0.5*\height) node[text width=2.3cm, xshift=1.3cm] {Allocation $\Algo'$ Payment $\Pay$};

\draw[->] (\x-1.5, 0.5*\y+0.5*\height) -- (\x, 0.5*\y + 0.5*\height) node[xshift=-2cm, yshift=0.1cm, text width = 2cm] { $\v': $ Values submitted };

\end{tikzpicture}

	\end{figure}
	\pause
	\begin{itemize}
		\item If $\Algo$ maximizes welfare \emph{exactly} in poly-time\\
			Implement the allocation of $\Algo$ + charge suitable payments \\
			$\rightarrow$ VCG is truthful! 
	{\footnotesize \textcolor{\refscolorz}{[Vickrey 1961, Clarke 1971, Groves	1973]}}



		\item If $\Algo$ only \emph{approximately} maximizes welfare\\
				\pause
				$\rightarrow$ VCG is not truthful! \\
        $\rightarrow$ There may not exist $\Pay$  such that $\Mech(\Algo,\Pay)$ is truthful.\\
$\rightarrow $ Goal: find a modified allocation $\Algo'$ and truthful payments?
	\end{itemize}	
	
	Is designing truthful $\Mech$ harder than the algorithmic problem? \\
	\pause
			 One possible answer: \textcolor{olive}{Black Box reductions!} \\


\end{frame}

\begin{frame}
	\frametitle{Black-Box Mechanism Design}

	\begin{tabular}{m{0.45\textwidth} m{0.55\textwidth}} 
		Someone gives us $\Algo$ that
		\begin{itemize}
			\item Returns feasible allocations
					\only<1,2,3>{ \item Has some welfare guarantee}
					\only<4,5,6>{	\item \textcolor{cyan}{Has some welfare guarantee}}
		\end{itemize}
	& \onslide<2,3,4,5,6>{
		\only<2>{	We implement $\Mech(\Algo', \Pay)$ s.t.}
		\only<3,4,5,6>{\textcolor{red}{Can we implement $\Mech(\Algo', \Pay)$ s.t \textbf{?}}}
		\begin{itemize}	
			\item Returns feasible allocations
					\only<2,3,4>{
			\item Have comparable welfare to $\Algo$
			}
\only<5,6>{\item \textcolor{violet}{Have comparable welfare to $\Algo$}
					}
					\only<2,3,4,5> {			\item Is incentive compatible (DSIC/BIC...)
					}
					\only<6>{			\item \textcolor{brown}{Is incentive compatible (DSIC/BIC...)}
					}
		\end{itemize}
		}
	\end{tabular}	
	
	%\textbf{Given}: Algorithm $\Algo$ returns feasible allocation $s$ to maximize social welfare $\sum_i v_i(s)$ \\

%	\textbf{Goal}: Turn $\Algo$ into mechanism $\Mech$ where agents are truthful. \\
%	$\rightarrow$ find $\Algo'$ and $\Pay$ s.t. $\Mech$ is truthful:
%	everyone satisfied by reporting their true value
\onslide<4,5,6>{
		\mbox{}\\
Many different flavours to the problem:		
		\begin{enumerate}
\onslide<4,5,6>{		\item \textcolor{cyan}{Worst Case performance \textbf{vs} Average Performance
				when $v_i \sim \Dist$} 
				}
				\onslide<5,6>{ 	
					\item \textcolor{violet}{Achieving $\Algo$'s welfare exactly \textbf{vs} approximately}
				}
			\onslide<6>{
			\item \textcolor{brown}{Truthfulness: DSIC \textbf{vs} BIC (\emph{Bayesian
					Incentive Compatible} $\rightarrow $ truthful in
					expectation over other agents reports)} 

			}
		\end{enumerate}

}

\end{frame}


\section{Previous and New Results}

\begin{frame}
	\frametitle{Previous Results}

	Can we find such a reduction from mechanism design to algorithm design? \\
	Flavours of the problem studied:\\
	\mbox{}\\

%	\begin{center}
%		Objective: \emph{Maximize Welfare}
%	\end{center}

	\begin{adjustbox}{center}
		\begin{tabular}{m{0.3\textwidth} m{0.8\textwidth}}
				           &   \mybullet Prior-Free Settings\\

%%%%%%%%%NEW LINE OF TABULAR %%%%%%%%%%%
			\ldelim\{{1}{2mm}[\parbox{32mm}{preserve \textbf{worst\\ case} approx.}] 
				 & 
			 \parbox{90mm}{ \sp \mybullet   Cannot find reduction to get \textbf{DSIC} Mechanism \\\spitem
					 even for single parameter\\\spitem 
	 				{\footnotesize \textcolor{\refscolorz}{[Chawla et al 2012]}}}\\

%%%%%%%%%NEW LINE OF TABULAR %%%%%%%%%%%
				 	& \\	
%%%%%%%%%NEW LINE OF TABULAR %%%%%%%%%%%

					& \mybullet  Bayesian Settings ($v_i \sim \Dist$)\\
%%%%%%%%%NEW LINE OF TABULAR %%%%%%%%%%%
			\ldelim\{{6}{100mm}[\parbox{30mm}{preserve \textbf{expected\\ welfare} within  $\varepsilon$} ]
					&
			\parbox{90mm}{ \sp \mybullet Can find \textbf{BIC} Mechanism, \textbf{single}-parameter\\ 
						    \spitem {\footnotesize \textcolor{\refscolorz}{[Hartline, Lucier 2010]}}}\\

%%%%%%%%%NEW LINE OF TABULAR %%%%%%%%%%%
				 	& \\	
%%%%%%%%%NEW LINE OF TABULAR %%%%%%%%%%%
				& \parbox{90mm}{ \sp \mybullet  Can find \textbf{$\epsilon$-BIC} Mechanism, \textbf{multi}-parameter\\
				\spitem {\footnotesize \textcolor{\refscolorz}{[Hartline et al 2011 and Bei, Huang 2011]}}}\\

%%%%%%%%%NEW LINE OF TABULAR %%%%%%%%%%%
				 	& \\	
%%%%%%%%%NEW LINE OF TABULAR %%%%%%%%%%%
				& \parbox{90mm}{  \sp \mybullet Can find \textbf{BIC} Mechanism, \textbf{multi}-parameter\\
				\spitem {\footnotesize \textcolor{\refscolorz}{[Dughmi et al 2017] }}}\\
	\end{tabular}
	\end{adjustbox}
\end{frame}


\begin{frame}
	\frametitle{What's left to do?}
	\begin{itemize}
		\item The picture so far

	\begin{itemize}
		\item \textcolor{red}{\xmark} DSIC reduction, worst-case performance, single-parameter
		\item \textcolor{green}{\checkmark} BIC reduction, expected performance, multi-parameter 
	\end{itemize}

	\pause
		\item Some questions still remain:
			\begin{enumerate}
				\item \emph{\textcolor{teal}{Can we find a ``stronger than BIC`` reduction that preserves
							expected welfare, even for single-parameter agents?}}
							\pause
				\item	Previous BIC results: runtime is polynomial in typespace* size.  \\
					$\rightarrow$ example: 
%					Poly-time is bad: e.g single
					additive agent, with independent values over each item,
					typespace is exponential.  \\
					\mbox{}\\
					
					\emph{\textcolor{teal}{Can we avoid runtime dependence on typespace?\\
					$\rightarrow$ get a BIC reduction that runs in time poly($n$,$m$)?}}\\
					
					  \mbox{}\\


			\end{enumerate}


			{\footnotesize \textcolor{gray}{*Typespace: \\
							discrete: possible different input profiles\\
							continuous: support size of $\Dist$} }\\
	\end{itemize}
 \end{frame}

\begin{frame}
	\frametitle{Main Results (Informal)}

%Two results: \\
%Progress towards answering previous questions:\\
\mbox{}\\
	\begin{itemize}\setlength\itemsep{1em}
		\item 
				\textcolor{red}{\xmark} No BIC reduction, even for single
				additive agent over independent items, with subexponential
				query complexity
		\item	 	 	
			\only<1>{\textcolor{red}{\xmark} No MIDR reduction even for single
					parameter settings, with subexponential query complexity\\
				}
				\only<2>{\textcolor{red}{\xmark} \textbf{No MIDR reduction even
						for single parameter settings, with subexponential
				query complexity}\\
			}
				\textcolor{gray}{{\small $\rightarrow$ MIDR $\subseteq$ DSIC $\subseteq $ BIC}}

	\end{itemize}

		\mbox{}\\
			\begin{center}
			\large{
				*\textcolor{red}{\xmark} =  \textbf{$\Mech$ degrades
				welfare by a polynomial factor} 
			} with subexponential queries to $\Algo$
 		\end{center}
		\onslide<2>{
		Up next: intuition for second result.}
\end{frame}




\section{Lower Bound Construction }
\begin{frame}
	\frametitle{Lower Bound for MIDR transformations}
	\begin{itemize}
		\item Objective: maximize welfare
		\item Single-parameter setting with $n$ agents
		\item For every agent: $v_i \in \{0,1\}$, outcome $\in \{0,1\}$
	\end{itemize}
	\mbox{} \\ 
	\begin{definition}[MIDR]
 $\Algo$ is MIDR if for every $\v,\v'$ :
  $
  \E_{\Algo(\bm{v})}[ \v \cdot \Algo(\v)] \ge \E_{\Algo(\bm{v'})} [\v \cdot \Algo(\v')]
   $ 
   \end{definition}
   Essentially: $\Algo(\bm{v})$ is best outcome for $\bm{v}$ in $\Algo$'s range.
	\pause
	
	\mbox{}\\
  
	\setcounter{theorem}{0}
		\begin{theorem}\label{thm:MIDR_formal}
	 For any MIDR black-box transformation $\Mech$ with sub exponential query complexity
	  there exists an algorithm $\Algo$ and
	  distribution $\Dist$ such that $\Wel(\Mech_{\Algo}) \le \frac{ \Wel(\Algo) }{\text{poly}(n)}$. 
	\end{theorem}
	
	\mbox{}\\


\end{frame}

\begin{frame}
	\frametitle{Construction Details}
	
	Construction:  family of algorithms $\Algo_{ST}$, distribution $\Dist$ such
	that $\Mech$ degrades welfare\\
	\mbox{}\\
	\begin{itemize}
	\setlength\itemsep{1em}
		\item Input distribution $\Dist$: $x_i=1$ w.p. $1/\O(\sqrt{n})$,
		\item Uniformly random hidden sets $S,T$ of size $\sim O(\sqrt{n})$\\
				with ``big enough`` intersection $|S\cap T|$
		\item Algorithm $\Algo_{ST}(x) = x \text{ or } 0$ depending on $x$\\
				Serve everyone with value $1$ \textbf{or} serve no one.\\
				Example:\\

				$\Algo_{S,T}(x) = x:\ 0010101 \rightarrow 0010101$ \textcolor{green}{\checkmark}
				$\Algo_{S,T}(x) = 0:\ 00\bm{1}0101 \rightarrow 00\bm{0}0000$ \textcolor{red}{\xmark}

	\end{itemize}
\end{frame}

\begin{frame}
		\frametitle{Illustration of Allocation}

		\begin{itemize}
				\item If $x$ is too large then $\Algo_{ST}(x)=\emptyset$
				\item If $x$ is not too large and has no intersection with $S$ and $T$, then $A_{ST}(x)=x$
		\end{itemize}

		\begin{figure}[!h]
			\centering
%			\resizebox{10cm}{5cm}{
		\input{cases_a_b.tex}
%			}
		\end{figure}

\end{frame}

\begin{frame}
		\frametitle{Illustration of Allocation}
If $x$ is not too large then:
		\begin{itemize}
				\item If  $|x\cap T|$ is large, and $|x\cap S|$ is small
						then $\Algo_{ST}(x)=\emptyset$
				\item else $\Algo_{ST}(x) = x$
		\end{itemize}

		\begin{figure}[!h]
			\centering
%			\resizebox{10cm}{5cm}{
		\begin{tikzpicture}[scale=0.80]
%Style macros
\tikzset{ST/.style={green, dashed, very thick}}
\tikzset{myset/.style={red, very thick}}
\tikzset{square/.style={black, very thick}}

%Circle macro, for rx and ry diameters
\tikzset{rx/.style={x radius=#1},ry/.style={y radius=#1}}

%Macros for graph parameters (length/height/radii)
\pgfmathsetmacro{\length}{6.5}
\pgfmathsetmacro{\height}{4}
\pgfmathsetmacro{\rad}{(\length-0.5)/4}
\pgfmathsetmacro{\rx}{2}
\pgfmathsetmacro{\ry}{5}

%Define array for labels
\def\lab{{"a","b","c", "d"}}

%Low left corners at (0,0), (0,-height-2), (length+2,0), (length+2, -height-2)
\def\xcoords{{0, \length+1,		    0,  \length+1}}
\def\ycoords{{0, 		 0,-\height-1, -\height-1}}

%%% MAIN CODE %%%

%Draw constant stuff (rectangles + sets S,T)
\foreach \i in {3,4} 
{ 

	\pgfmathsetmacro{\labelz}{\lab[\i-1]}
	\pgfmathsetmacro{\x}{\xcoords[\i-1]}
	\pgfmathsetmacro{\y}{\ycoords[\i-1]}

	%Draw rectangle, label at (length/2, \y-0.5)
	\draw[square] (\x, \y) rectangle (\x + \length,\y + \height) node[] at (\x + \length/2,\y - 0.5){(\labelz)};
	
	%Draw sets S and T
	\draw[ST] (\x + \rad + 1,\y + 1) circle [rx=\rad cm, ry=0.5cm] node[black, left=1.2cm]{$S$};
	\draw[ST] (\x + \length - \rad -1,\y + 1) circle [rx=\rad cm, ry=0.5cm] node[black, right=1.2cm]{$T$};

}

%Draw custom stuff for each rectangle
%Rectangle c
\pgfmathsetmacro{\x}{\xcoords[2]}
\pgfmathsetmacro{\y}{\ycoords[2]}
\draw[myset] (\x + 2*\length/3, \y+2) circle[rx = 0.6cm, ry = \rad cm];
\node[black] at (\x+2, \y+\height-1) {$|x\cap T|$ large}; % > \epsilon_T N$};
\node[black] at (\x+2, \y+\height-1.5) {$|x\cap S|$ small};% < \epsilon_S N$};
\node[red] at (\x+0.9*\length, \y + 0.9*\height) {\Huge \xmark};

%Rectangle d
\pgfmathsetmacro{\x}{\xcoords[3]}
\pgfmathsetmacro{\y}{\ycoords[3]}
\draw[myset] (\x + \length/2, \y + 1.5) circle[rx = 0.9cm, ry = \rad cm, rotate=-60];
\node[black] at (\x+2, \y+\height-1) {$|x\cap T| $ large}; %> \epsilon_T N$};
\node[black] at (\x+2, \y+\height-1.5) {$|x\cap S|$ large};% \geq  \epsilon_S N$};
\node[green] at (\x+0.9*\length, \y + 0.9*\height) {\Huge \checkmark};

\end{tikzpicture}

%			}
		\end{figure}

\end{frame}

\begin{frame}
	\frametitle{Lower Bound - Proof Idea}
  \begin{restatable}{lem}{welf_LB1}
	  \label{lem:welf_LB1}
	  $\Algo$ has high expected welfare ($\Omega(\sqrt{n})$) 
	\end{restatable}
	\onslide<2>{
	\begin{proofid}
			$\Algo_{ST}(x) = x$ almost always, result follows from concentration
			using construction of $S$, $T$ and $\Dist$
	\end{proofid}
}
  \begin{restatable}{lem}{welfUB}
  \label{lem:welf_UB1}
  $\Mech$ has polynomially lower welfare  than $\Algo$ ($O(n^{1/4})$)	
	\end{restatable}
	
	\mbox{}\\
	Lemma \ref{lem:welf_LB1} + Lemma \ref{lem:welf_UB1} $\rightarrow$ Theorem \ref{thm:MIDR_formal} 

\end{frame}

\begin{frame}
	\frametitle{Lower Bound - Proof Idea}
  \welfUB*
	
  \begin{proofid}\renewcommand{\qedsymbol}{}
		Prove in 3 steps:
		\begin{tabular}{m{5cm} m{6cm}}
			\begin{enumerate}
				\item $\Algo_{ST}(T) = 0$, and $\Mech$ cannot find set S with
						subexponentially many samples, so it \textbf{cannot find an
						output with high welfare for $T$}.

				% \item  $\Mech_\Algo(T)$ gives low welfare \textbf{Idea}:
				% 		$\Algo_{ST}(T)=0$, $\Mech$ cannot find set	$S$ to
				% 		return something large 
			\end{enumerate}
		& 
			\begin{center}
				\begin{minipage}{.3\textwidth}
				 	\resizebox{4cm}{3cm}{
						\input{case_c.tex}
					}
				\end{minipage}
			\end{center}
		\end{tabular}
	\end{proofid}
	Note: we don't use any truthfulness constraint here
\end{frame}


\begin{frame}
	\frametitle{Lower Bound - Proof Idea}
	\welfUB*
	
	\begin{proofid}
		\begin{tabular}{m{4.8cm} m{4.5cm}} 
			\begin{enumerate}[2.]
				\item \textbf{Idea}:	Because of MIDR, $\Mech_{\Algo}(S)$
						can't return an outcome with high welfare for $T$.
						%t allocate many items in $S \cap T$. 
						However, on input $S$, we cannot find $T$ $\rightarrow$
						must reduce welfare throughout.
			\end{enumerate}
			& 	
			\begin{center}
				\begin{minipage}{.3\textwidth}
				 	\resizebox{3.5cm}{2.5cm}{
						\input{case_d.tex}
					}
				\end{minipage}
			\end{center}\\
\pause
			\begin{enumerate}[3.]
				\item $\Mech_\Algo(x)$ gives low welfare (for any input $x$)
		 			\textbf{Idea}: Cannot decide if $x$ is the set $S$ or not
			\end{enumerate}
			&
			\begin{center}
				\begin{minipage}{.3\textwidth}
				 	\resizebox{5.5cm}{2.5cm}{
						\input{case_b.tex}
						\input{case_d.tex}
					}
				\end{minipage}
			\end{center}
	
		\end{tabular}

%	$\Rightarrow$ \textbf{Always} give low welfare!
\end{proofid}
\end{frame}


 \begin{frame}
 \frametitle{Second Result}

 \begin{itemize}
		\item Objective: maximize $\Wel$
		\item Single additive agent, $n$ items
%		\item Type space $\Domain = \{0,1, \alpha\}^n$
%		\item Outcome space $\Range = \{0,1\}^n$
	\end{itemize}

		\begin{theorem}\label{thm:BIC_formal}
			For any DSIC black-box transformation $\Mech$ with sub exponential query
			complexity, there exists an algorithm $\Algo$ and distribution $\Dist$ such
			that $\Wel(\Mech_{\Algo}) \le \frac{ \Wel(\Algo) }{\text{poly}(n)}$
	\end{theorem}
	\mbox{}\\
	Note: for single agent, DSIC = BIC $\Rightarrow$ same result for BIC\\
	\mbox{}\\ 

	Proof follows similarly to MIDR reduction, \textbf{but}
	\begin{itemize}
%		\item Change type space to $\Domain =\{0,1,\alpha\}^n$ 
		\item Instead of MIDR condition, uses a characterization of BIC allocation rules due to 
%		\item Use characterization of allocation rules by
				{\footnotesize
				\textcolor{\refscolorz}{[Hartline, Kleinberg, Malekian 2011]}}
	\end{itemize}
		
 \end{frame}

\section{Conclusion - Open Problems}
\begin{frame}
	\frametitle{Conclusion and Open Problems}

	Black-box reductions: \\
	\begin{center}
			reducing mechanism design to algorithm design.
	\end{center}

 	Existing black-box reductions are for BIC mechanisms, and have polynomial dependence on typespace\\

	%Black-box reductions not possible without priors\\
%	Black-Box reductions with priors require \emph{BIC} + \emph{Polynomial dependence on typespace}\\
	\mbox{}\\
	We showed two negative results
	\mbox{}\\

	\begin{itemize}
	\setlength\itemsep{1em}	
	\item Remove polynomial dependence on typespace \\
					\textcolor{red}{\xmark} Even for single additive agent over independent types

		\item Change BIC requirement $\rightarrow$ strengthen BIC to MIDR\\
			\textcolor{red}{\xmark} Even for single parameter settings
		\end{itemize}
	\mbox{}\\
	\onslide<2,3>{
	\textbf{Open problem}: Can we strengthen BIC to DSIC, while preserving
	expected welfare? {\footnotesize \textcolor{\refscolorz}{[Chawla et al
	2012]}}
	}
		\mbox{}\\
	\pause
	\onslide<3>{
		\begin{center}
		{\large 	\textbf{Thank you!} }
	\end{center}
	}
\end{frame}


\end{document}
